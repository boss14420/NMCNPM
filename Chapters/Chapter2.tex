\chapter{Quy trình phát triển phần mềm}
    \section{Định nghĩa}
        Toàn bộ quy trình quản lý phát triển phần mềm gắn với khái niệm vòng đời phần mềm, được mô hình hóa với những kỹ thuật và phương pháp luận trở thành các chủ đề khác nhau trong Công nghệ phần mềm.

        \paragraph{Vòng đời của phần mềm} là thời kì tính từ khi phần mêm được
        bắt đầu sinh ra cho đến khi chết đi (từ lúc hình thành đáp ứng yêu cầu,
        vận hành, bảo dưỡng cho đến khi loại bỏ không dùng nữa),

        \paragraph{Chuẩn ISO/IEC 12207:2008} là một chuẩn quốc tế về quy trình phát triển phần
        mềm, định nghĩa tất cả các công việc cần thiết cho việc phát triển và
        bảo trì phần mềm \cite{ISOWiki}.

        Chuẩn ISO/IEC 12207:2008 quy định 43 quá trình (process), trong đó có
        các pha chính là: 
            \begin{itemize}
                \item phân tích/xác định yêu cầu người dùng, 
                \item thiết kế, 
                \item mã hoá, 
                \item kiểm thử, 
                \item bảo trì.
            \end{itemize}

        \paragraph{Mô hình phát triển phần mềm} sắp xếp các quá trình phát
        triển phần mềm theo một cách nào đó. Có nhiều mô hình phát triển phần
        mềm khác nhau, mỗi cái đều có ưu và nhược điểm riêng. Có thể kể đến:
        \begin{itemize}
            \item Mô hình thác nước (Waterfall model),
            \item Mô hình xoắn ốc (Spiral model),
            \item Mô hình lập trình linh hoạt (agile development),
            \item Mô hình chế thử (prototype model),
            \item \ldots
        \end{itemize}

    \section{Các quá trình trong phát triển phần mềm}
        \subsection{Phân tích - xác định yêu cầu}
        
        \subsection{Thiết kế}

    \section{Một số mô hình phát triển phần mềm}
        \subsection{Mô hình thác nước}
